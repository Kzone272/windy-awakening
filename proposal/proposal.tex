\documentclass {article}
\usepackage{fullpage}

\begin{document}

~\vfill
\begin{center}
\Large

A5 Project Proposal

Title: Windy Awakening

Name: Kevin Haslett

Student ID: 20468033

User ID: kahaslet
\end{center}
\vfill ~\vfill~
\newpage
\noindent{\Large \bf Final Project:}
\begin{description}
\item[Purpose]:\\
	To create a simple game that attempts to replicate the art style of The Legend of Zelda: Wind Waker.

\item[Statement]:\\

	Paragraph: What it's about.
The goal of this project to create somewhat of an homage to one of my favourite games.  In particular, I have always loved the art style of Wind Waker, and this project will be an attempt to recreate a number of the most iconic elements of this game.  One of the largest elements of the game, both mechanically, and thematically, is the sailing around the ocean from destination to destination.  Of course another key part of the game is defeating dungeons, but that will be out of the scope for the project.  Because of this I will be focusing solely on creating the nice peaceful sailing experience.

	Paragraph: What to do.
Easily the most iconic visual element of this game is the cell-shaded art style. There is also a very distinctive water texture that needs to be recreated to simulate the look of waves.  Of course in order to convey object detail, we will also need texture mapping.  And shadows are another simple element that adds a lot of depth.  And since Wind Waker is a very bright and colourful game, a bloom filter adds a nice it of polish.  I'd also like to implement a day-night cycle which allows the lighting to change over time.  We'll also need some islands scattered throughout the sea to add some variation amidst the sea, but for the sake of simplicity we will bound the player to their boat, so we don't have to add much detail to the islands.  And finally we'll have some simple boating physics to make the sailing experience somewhat believable.

	Paragraph: Why it is interesting and challenging.

	Paragraph: What I will learn

\item[Technical Outline]:\\
    Basically, your objectives in your objective list should be fairly
    short statements of the objective; you should provide additional
    details about your objectives in this section to clarify what you
    plan to do.

     Further, survey the important data structures and algorithms that
     will be necessary to achieve the goals, and (for ray tracing
     projects) lists the new commands
     that will need to be added to the input language.

     To  get  bold face: {\bf bold face words}.  To get italics: {\it italic
     face words}.  To  get typewriter font: {\tt typed words}.  To get
     larger  words:  {\large large  words}.   To  get smaller words: 
     {\small small words}.  

\item[Bibliography]:\\
     Articles  and/or  books  with  important  information on the
     topics of the project.

\end{description}
\newpage


\noindent{\Large\bf Objectives:}

{\bf Full UserID: kahaslet\hfill{\bf Student ID: 20468033}

\begin{enumerate}
     \item[\_\_\_ 1:]  Texture Mapping

     \item[\_\_\_ 2:]  Cell Shading

     \item[\_\_\_ 3:]  Shadow Maps

     \item[\_\_\_ 4:]  Bloom Shader

     \item[\_\_\_ 5:]  Day Night Cycle

     \item[\_\_\_ 6:]  Voronoi Diagrams

     \item[\_\_\_ 7:]  Edge Detection

     \item[\_\_\_ 8:]  Guassian Blur and Processing to Generate Final Water Texture

     \item[\_\_\_ 9:]  Island Terrain Generated Using Perlin Noise

     \item[\_\_\_ 10:]  Simple Sailing Physics
\end{enumerate}

% Delete % at start of next line if this is a ray tracing project
% A4 extra objective:
\end{document}
